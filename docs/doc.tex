\documentclass[12pt]{article}
\usepackage[backend=bibtex,style=numeric]{biblatex} 
\addbibresource{references.bib} 

\title{Report for the CPS Mid-Course Project}
\author{
    Andrea Auletta \\ \texttt{andrea.auletta@studenti.unipd.it} \and
    Niccolò Zenaro \\ \texttt{niccolo.zenaro@studenti.unipd.it}
}
\begin{document}

\maketitle
\newpage
\tableofcontents
\newpage

\section{Introduction and objectives}
In this report we will describe the work done for the mid-course project of the course Cyber-Physical Systems and IoT security. 
The paper to which we refer is \textbf{\cite{Cho2016} \citetitle{Cho2016}} where is explained the design of an algorithm: the CIDS (Clock-based Intrusion Detection System), that is able to detect different kinds of attack. In this paper is shown the functioning of three kinds of attack: the "Fabbrication attack", the "Suspension attack" and the "Masquerade attack". What we tried to do was to implement the three attacks and the CIDS algorithm in a simulation environemnt using \textbf{ICSim}, instead of physically as shown in the paper.
\section{System setup}
We worked on a virtual machine with \textbf{LinuxMint22} as operating system.
The simulator used is \textbf{ICSim}. All the code is written in \textbf{Python} and can be found at the following link: .% \url{}
We used the \textbf{can library} to create messages and to simulate the communication of the attackers with the ECUs.
\section{Experiments}
\subsection{Fabbrication attack}
\subsection{Suspension attack}
\subsection{Masquerade attack}
\subsection{CIDS algorithm}
\section{Results and Discussion}
\printbibliography 

\end{document}
